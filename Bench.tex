\documentclass{article}
\newtheorem{hypothesis}{Hypothesis}
\usepackage{graphicx}
\usepackage{url}
\bibliographystyle{plain}
\begin{document}
\title{Benchmarking}
\author{MJB/JHD/AG}
\maketitle
\begin{abstract}\noindent
The SAT community, and hence the SMT community, have substantial experience in benchmarking solvers against each other on large sample sets, and publishing summaries, whereas the computer algebra community tends to time solvers on a small set of problems, and publishing individual times.
\par
This paper aims to document the SAT community practice for the benefit of the computer algebra community.
\end{abstract}
\input BenchInner
\section{Virtual Best Solver}
The SAT competition has taken to including a "virtual best solver" (VBS)
which is synthesised from the other results by taking the minimum (across all solvers tested) time taken to solve every given benchmark.
Thus the VBS is always equal to the time of some solver, but which one will change by the benchmark (measuring how often each solver is the VBS is also an interesting metric).  The VBS can be added to the survivor/cactus plot to get a feeling for the variability between solvers.
\par
We have therefore added this to our solvers on the diagrams.
\section{Case Studies}
\begin{figure}[h]
\caption{Data from Section \ref{sec:Pr}\label{Fig:S2}}
\includegraphics[scale=0.25]{Fig2a.jpg}
\includegraphics[scale=0.25]{Fig2b.jpg}
\end{figure}
\subsection{Predictable}\label{sec:Pr}
We used a vector of baseline times (notionally in seconds) of \\\verb+cat(2,[1.1:0.002:2],[2:1:50],[50:5:300]);+ in MatLab speak, i.e. 1.1 to 2 in steps of 0.002, 2 to 50 in step of 1, and 50 to 300 in steps of 5. We first measure four solvers: baseline, baseline less 1 second, 40\% of baseline and a hybrid of 70\% of (baseline less 0.5 seconds). The results are shown in Figure \ref{Fig:S2}. ``1 second faster'' was quickest 284 times, and ``60\% faster'' 267 times. However, ``60\% faster'' took 48.3 seconds longer than the Virtual Best (which took 4311 seconds),``1 second faster'' 6036 seconds, hybrid  3125 seconds and the baseline 6572 seconds.
\subsection{Predictable plus Fuzz}\label{sec:P+f}
\begin{figure}[h]
\caption{Data from Section \ref{sec:P+f}\label{Fig:S3}}
\includegraphics[scale=0.60]{Fig3a.jpg}
\includegraphics[scale=0.60]{Fig3b.jpg}
\end{figure}
What happens if we multiply each time by a random variable, uniform in [0.8,1.2]? 
%``1 second faster'' was quickest 292 times, hybrid 8 times and ``60\% faster'' 251 times.
40 runs of this experiment give a mean VBS time of 4299 seconds, with a standard deviation of 53.455. In the counts of how often each solver was VBS, hybrid appeared, showing up between 5 and 17 times, with corresponding adjustments to the others. ``1 second faster'' was always the most common, with the ratio of it over ``60\% faster'' ranging from 1.09 to 1.34. The plots (linear and semilogx) are in Figure \ref{Fig:S3}.
\subsection{Predictable plus Random}\label{sec:P+J}
\begin{figure}[h]
\caption{Data from Section \ref{sec:P+J}\label{Fig:S4}}
\includegraphics[scale=0.25]{Fig4a.jpg}
\includegraphics[scale=0.25]{Fig4b.jpg}
\end{figure}
To the previous solvers, we add a ``joker'', that, on one problem in 10, takes 10\% as long as the baseline.  The results are shown in Figure \ref{Fig:S4}. The joker was quickest 55 times, ``1 second faster'' was quickest 256 times, and ``60\% faster'' 240 times. The time differences are that ``60\% faster'' took 402 seconds longer than the Virtual Best, ``1 second faster''  6390 seconds, hybrid 3479 seconds, the joker 5865 seconds and the baseline 6941 seconds.
%\subsection{Predictable}
%\subsection{Predictable}
\section{Martin Brain adds}
\begin{enumerate}
%\item
%The SAT competition has taken to including a "virtual best solver"
%which is synthesised from the other results by taking the minimum (across all solvers tested) time taken to solve every given benchmark.
%Thus the VBS is always equal to the time of some solver, but which one will change by the benchmark (measuring how often each solver is the VBS is also an %interesting metric).  The VBS can be added to the cactus plot to get a feeling for the variability between solvers.
\item
Scatter plots are used to compare pairs of solvers.  For each benchmark you plot (sometimes using different colours or marks for SAT and UNSAT) a point with x location the time taken by solver 1 and y the time taken by solver 2.  To make things easier to follow, people commonly add the diagonal (sometimes annotated with "solver 1 is faster"
and "solver 2 is faster" on the relevant sides / corners) and the time-out lines.
\end{enumerate}
\bibliography{../../../../../../jhd}
\end{document}
\begin{figure}
\caption{}
\includegraphics{}
\end{figure}